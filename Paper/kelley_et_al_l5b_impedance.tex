%%%%%%%%%%%%%%%%%%%%%%%%%%%%%%%%%%%%%%%%%%%%%%%%%%%%%%%%%%%%%%%%%%%%%%%%%%%%%%%%%%%%%%%%%%%%%%%%%%%%%%%%%%%%%%%%%%%%%%%%%%%%%%%%%%%%%%%%%%%%%%%%%%%%%%%%%%%
% When submitting your files, remember to upload this *tex file, the pdf generated with it, the *bib file (if bibliography is not within the *tex) and all the figures.
%%%%%%%%%%%%%%%%%%%%%%%%%%%%%%%%%%%%%%%%%%%%%%%%%%%%%%%%%%%%%%%%%%%%%%%%%%%%%%%%%%%%%%%%%%%%%%%%%%%%%%%%%%%%%%%%%%%%%%%%%%%%%%%%%%%%%%%%%%%%%%%%%%%%%%%%%%%

%%% Version 3.4 Generated 2018/06/15 %%%
%%% You will need to have the following packages installed: datetime, fmtcount, etoolbox, fcprefix, which are normally inlcuded in WinEdt. %%%
%%% In http://www.ctan.org/ you can find the packages and how to install them, if necessary. %%%
%%%  NB logo1.jpg is required in the path in order to correctly compile front page header %%%

\documentclass[utf8]{frontiersSCNS} % for Science, Engineering and Humanities and Social Sciences articles
%\documentclass[utf8]{frontiersHLTH} % for Health articles
%\documentclass[utf8]{frontiersFPHY} % for Physics and Applied Mathematics and Statistics articles
\usepackage[english]{babel}
\selectlanguage{english}
%\setcitestyle{square} % for Physics and Applied Mathematics and Statistics articles
\usepackage{url,hyperref,lineno,microtype,subcaption}
\usepackage[onehalfspacing]{setspace}

\linenumbers

% Leave a blank line between paragraphs instead of using \\
\def\keyFont{\fontsize{8}{11}\helveticabold }
\def\firstAuthorLast{Kelley {et~al.}} %use et al only if is more than 1 author
\def\Authors{Craig Kelley\,$^{1,*}$, Salvador Dura-Beranal\,$^{2,3}$, Samuel A. Neymotin\,$^{3}$, Nicholas T. Carnevale\,$^{4}$, and William W. Lytton\,$^{2,5,6,7}$}
% Affiliations should be keyed to the author's name with superscript numbers and be listed as follows: Laboratory, Institute, Department, Organization, City, State abbreviation (USA, Canada, Australia), and Country (without detailed address information such as city zip codes or street names).
% If one of the authors has a change of address, list the new address below the correspondence details using a superscript symbol and use the same symbol to indicate the author in the author list.
\def\Address{$^{1}$Program in Biomedical Engineering, SUNY Downstate Health Sciences University \& NYU Tandon School of Engineering, Brooklyn, NY, USA \\
$^{2}$Department of Physiology \& Pharmacology, SUNY Downstate Health Sciences University, Brooklyn, NY, USA\\
$^{3}$Center for Biomedical Imaging and Neuromodulation, Nathan S. Kline Institute for Psychiatric Research, Orangeburg, USA\\
$^{4}$Department Neuroscience, Yale University, New Haven, USA\\
$^{5}$Department of Neurology, SUNY Downstate Medical Center, Brooklyn, NY, USA\\
$^{6}$Department of Neurology, Kings County Hospital Center, Brooklyn, NY, USA\\
$^{7}$The Robert F. Furchgott Center for Neural and Behavioral Science, Brooklyn, NY, USA}
% The Corresponding Author should be marked with an asterisk
% Provide the exact contact address (this time including street name and city zip code) and email of the corresponding author
\def\corrAuthor{Corresponding Author}
\def\corrEmail{craig.kelley@downstate.edu}

\begin{document}
\onecolumn
\firstpage{1}

\title[Impedance Profiles of L5 Pyramidal Neurons]{Complex Impedance Profiles of Neocortical Layer 5 Pyramidal Neurons} 

\author[\firstAuthorLast ]{\Authors} %This field will be automatically populated
\address{} %This field will be automatically populated
\correspondance{} %This field will be automatically populated
\extraAuth{}% If there are more than 1 corresponding author, comment this line and uncomment the next one.
%\extraAuth{corresponding Author2 \\ Laboratory X2, Institute X2, Department X2, Organization X2, Street X2, City X2 , State XX2 (only USA, Canada and Australia), Zip Code2, X2 Country X2, email2@uni2.edu}
\maketitle
\begin{abstract}
%%% Leave the Abstract empty if your article does not require one, please see the Summary Table for full details.
\section{}
Pyramidal neurons in neocortex have complex input-output relationships that depend on their morhpologies, active and
passive channel distributions, and the nature of their inputs, and which cannot be replicated by their simple 
integrate-and-fire analogs. Measures of the complex impedance response across neurons' dendritic arbors, such as 
resonance and the phase response, can provide intraneuronal functional maps reflecting their intrinsic dynamics and excitability. 
Experimental studies of dendritic impedance show that pyramidal neurons in hippocampal CA1 and neocrotex exhibit distance-dependent
relationships for the resonance and the phase responses with respect to the soma. We present a detailed study of
the complex impedance profiles of biophysically detailed models of neocortical layer 5 pyramidal neurons from 
across neocortex.  While none of these models were designed to fit the impedance response directly, two closely 
recapitulate the distance-dependent relationships of the complex impedance response observed experimentally. The  impedance response is also
dynamically tunable via the activity-dependent expression of voltage-gated ion channels in the dendrites, like HCN channels
and M- and A-type K+ channels. We investigated the roles of different voltage-gated ion channels in tuning the impedance response,
showing that changes in HCN and M-type K+ channel conductances, both locally and across the dendritic arbor, can modulate the
impedance response, producing a wide range of profiles.  We also clearly demonstrate the impedance profile's dependence on cell
morphology.  Although a number of cell models exhibit impedance profiles that diverge from the experimental data, some are
biologically plausible and may reflect regional variation or cells with different histories and therefore
differently tuned impedance responses. Given the dearth of experimental data on dendritic impedance in neocortical pyrmaidal neurons 
from different cortical regions, we hypothesize that neocortical pyramidal neurons have more diverse impedance profiles 
than previously assumed.

\tiny
 \keyFont{ \section{Keywords:} Pyramidal neuron, Impedance, Neocortex, Resonance, Phase Response, h-current, Inductive Reactance} %All article types: you may provide up to 8 keywords; at least 5 are mandatory.
\end{abstract}

\section{Introduction}
% Overview of Neocortical Pyramidal Cells
The pyramidal cells (PCs) found in layer 5 (L5) of neocortex are the main outputs of the cortex and project to various cortical 
and subcortical structures, signaling their role in top down control of other brain areas and 
motor function \citep{Levesque1996-hf, Veinante2000-gg, Hattox2007-km, Aronoff2010-do, Harris2015-te, Naka2016-rn}.
In order to produce their outputs, L5 PCs integrate inputs from all other cortical layers and thalamus \citep{Agmon1992-mi, 
Meyer2010-pe, Wimmer2010-lq, Oberlaender2012-fh, Rah2013-kr, Markram2015-zg}. There is great diversity among PCs in L5, not just
in their  morphologies and projections, but also in their spiking activity, with some PCs having high spontaneous firing rates while 
others' firing rates are closely correlated with the activity of neurons in the surrounding population \citep{Markram2015-zg}.
The balance of excitatory and inhibitory inputs and the electrotonic structure of PCs are key in understanding how PCs generate their
outputs and exert top-down control over other parts of the nervous system.  PCs are not, however, the only major output neurons in L5.
There are also intertelencephalic (IT) or commisural neurons, which project to contralateral cortex.  One of the main factors, besides their
projections, that distinguish PCs from IT cells is their comparatively high expression of the hyperpolarization-activated cyclic 
nucleotide–gated (HCN) channel, a nonselective voltage-gated cation channel responsible for the h-current (I$_h$) \citep{Oswald2013-lh}.

% Overview of I$_h$
I$_h$ has some dramatic effects on the intrinsic dynamics and exictability of neurons.  It acts as a pacemaker current, supporting burst
and regular firing modes.  It mediates the sag potential observed during hyperpolarization and spike frequency adaptation during
suprathreshold depolarization \citep{Robinson2003-uc, Oswald2013-lh}. I$_h$ also mediates coincidence detection \citep{Das2015-mh, Dewell2019-ra} 
and has been suggested to determine the frequency response of neuronal membrane potential in response to weak 
alternating electric field stimulation \citep{Toloza2018-vh}.  HCN channels have also been shown to have paradoxical effects on 
excitatory post-synaptic potentials (EPSPs), enhancing spiking in response to EPSPs when the spike threshold is low and inhibiting 
spiking in response to EPSPs when the spike threshold is low \citep{George2009-ad}.  This may implicate a role for HCN channels in the 
separation of planning and action.  Due to the high expression of HCN channels in PCs, PCs respond to dynamic stimuli as bandpass filters, 
whereas IT cells behave like low-pass filters \citep{Dembrow2010-lb}.

% Intro to impedance
The filtering properties of the subthreshold voltage response can be characterized by the complex impedance profile of the neuron.
A common experimental method for probing the complex impedance response is to stimulate the neuron with a chirp current injection, which
is a constant-amplitude, sinusoidal waveform whose instantaneous frequency is a linear function of time.  The input impedance:
\begin{equation}  Z_i{_n} = \frac{FFT(V_m)}{FFT(I_i{_n})} \label{eq:01}\end{equation}
is a complex valued function where FFT(I$_i{_n}$) is the Fourier transform of the injected current waveform and FFT(V$_m$) is the Fourier
transform of the local change in membrane potential.  Similarly, the transfer impedance (Z$_c$) can be calculated between the site of stimulation
and any point on the neuron by changing V$_m$ for some distal membrane voltage recording V$_s$ in \emph{Eq. 1}. From the impedance we can
extract the real valued resitance ($R$) and the imaginary valued reactance ($X$).  From $R$ and $X$ we can calculate the impedance amplitude:
\begin{equation} |Z| = \sqrt{R^2 + X^2} \label{eq:02}\end{equation}
A peak in the $|Z|$ profile at some nonzero frequency is called resonance, and it is driven by
the inductive reactances imparted by voltage-gated ion channels like HCN, M-type K+, and 
K-type K+ channels \citep{Das2017-nz}.  The resonance strength can be quantified by the 
Q-factor, the ratio of resonant $|Z|$ to the initial $|Z|$ \citep{Dewell2019-ra}. The 
resonance frequency specifies the frequency at which $|Z_{in}|$ is maximized, while
transfer frequency specifies the frequency at which $|Z_c|$ between the stimulation site and 
the soma is maximized \cite{Dembrow2015-zb}.  Subthrehold resonance has been observed in a wide 
variety of species and neuronal cell types \citep{Crawford1981-av, Puil1988-ca, Hutcheon2000-gs, Yoshida2011-ec, Ulrich2002-dd},
and is proposed to impart neurons with the ability to discriminate their inputs by their 
frequency content \citep{Das2017-nz}. Furthermore, a direct correlation has been shown between subthreshold resonance and suprathrehold
resonance via spike-triggered averaging methods \citep{Das2015-mh}.
Meanwhile, impedance phase ($\Phi$) defines the temporal relationship between I$_{in}$ and membrane potential:
\begin{equation} \Phi = arctan(\frac{X}{R}) \label{eq:03}\end{equation}
When $\Phi$ $>$ 0, the changes in membrane potential lead  I$_{in}$, which is referred to as leading phase.
When $\Phi$ = 0, I$_{in}$ and membrane potential are synchronous, and otherwise I$_{in}$ leads the membrane potential.
Leading phase is driven by the balance of capacitive and inductive reactances and is thought to provide a potential 
mechanism by which subthreshold neuronal membrane oscillations might maintain phase relationships with ongoing
local field potentials
\citep{Mauro1961-ys, Sabah1969-at, Mauro1970-km, Hu2002-ga, Hu2009-qb, Ulrich2002-dd, Cook2007-cz, Narayanan2008-zw, Vaidya2013-sx, Das2017-nz}.  
By determining the various properties of the impedance response across the dendritic arbor, we can produce intraneuronal 
functional maps which characterize the neuron's intrinsic dynamics and excitability \citep{Narayanan2012-hn}.

%% location dependence
A fundamental property of the impedance profiles of PCs is location-dependence. Experimental
studies show relationships between dendritic resonance, resonance strength, and 
phase response with distance from the soma \citep{Das2017-nz, Narayanan2007-gw, Ulrich2002-dd, Dembrow2015-zb}.
The earliest observations of location-dependence in neocrotical PCs from somatosensory cortex 
showed transfer impedance amplitude between the dendrite and soma and resonance strength 
increasing roughly linearly with distance the soma \citep{Ulrich2002-dd}. Further evidence for 
location-dependence of resonance frequency and synchronous frequency in PCs from prefrontal cortex \citep{Dembrow2015-zb}.
In both aforementioned studies, resonance and synchronous frequencies were found in the theta 
range (3-12 Hz), but also sampled a fairly small fraction of the dendritic arbor (120-280 um and 
200-600 um respectively) \citep{Ulrich2002-dd, Dembrow2015-zb}.

%% tunability and degeneracy
Impedance profiles are note static.  There is overwhelming evidence that subthreshold resonance
properties are dynamically tunable across the entire dendritic arbor \citep{Magee2005-oq, Narayanan2007-gw, Narayanan2008-zw, Sjostrom2008-sz, Hu2009-qb, Rathour2012-am, Rathour2012-bu, Das2017-nz}.
For instance, LTP induces changes in the impedance profile of hippocampal PCs \citep{Narayanan2007-gw}.
While resonance is mediated by the HCN channel, it can be significantly modulated by other voltage-gated ion channel conductances, (like I$_m$ from M-type K$^+$ channels)
and morphological changes \citep{Hutcheon2000-gs, Hu2002-ga, Narayanan2008-zw, Zemankovics2010-zt, Rathour2012-bu, Dhupia2014-ad, Rathour2016-vv}
At the same time, the impedance profile exhibits degeneracy: the same resonance properties can be arrived at by
multiple combinations of channel expressions \citep{Rathour2012-bu, Rathour2014-pl, Das2017-nz}. While dynamic 
tunability allows the impedance profile to change in response to external influences, degeneracy can maintain the 
impedance profile in the presence of external influences.  Dynamic changes to impedance profiles may have a role in
pathophysiology, with I$_h$ being significantly altered in temporal lobe epilepsy \citep{Shin2008-za, Marcelin2009-vy}.

\section{Methods}
% where you can find models and code used in this study
Four of the models studied in this paper \citep{Kole2008-aj, Acker2009-yj, Hay2011-if, Neymotin2017-dr} are
are available for download on ModelDB (\url{https://senselab.med.yale.edu/modeldb/}).  The fifth is an "all-active"
biophysically detailed model available from the Allen Brain Atlas Data Portal (\url{https://portal.brain-map.org/}) \citep{Reimann2013-mg, Shai2015-ff, Markram2015-zg}.
All simulations presented here were performed using NEURON in Python \citep{Hines2009-qx}.  The code 
developed for simulation, data analysis, and visualization were written in Python and MATLAB (Natick, Ma), and it
is available on GitHub (https://github.com/suny-downstate-medical-center/L5PYR\_Resonance).

% overview of the models
We focus our study on three models of rat PCs and three models of mouse PCs. A summary of the basic
model information is provided in \label{Table 1}.
% Hay 2011
The first model \citep{Hay2011-if}, is based on data from neocortex of Wistar rats, P36.  The model was fit to perisomatic
and backpropagating spiking activity.  In the model, all dendritic channels are uniformly distributed with the 
exception of I$_h$.  I$_h$ is uniform in the basal dendrites, while in the apical dendrites I$_h$ channels are
distributed using a density function that increases exponentially with distance from the soma 
\citep{Nevian2007-gw, Kole2006-bm}.
% Acker Antic 2009
The second model \citep{Acker2009-yj}, is based on data from frontal cortex of 
Sprague-Dawley rats, P21-33.  The model was fit using voltage-sensitive dye imaging data with a focus on 
backpropagating action potentials in the basal dendrites.
% Kole 2008
The third rat PC models \citep{Kole2008-aj} is based on data from somatosensory cortex of Wistar rats. Channel 
densities were adjusted  primarily to account for perisomatic spiking activity, particularly fast action potential repolarization and 
large amplitude  afterhyperpolarization in the axon initial segment.  I$_h$ channels are distributed throughout 
the dendritic arbor with an exponential increase in density with distance from the soma \citep{Kole2006-bm}.
It also has I$_m$ channels distributed evenly throughout the dendritic arbor.
% Neymotin 2017
The first mouse model \citep{Neymotin2017-dr}, is based on data from primary motor cortex (M1) of C57Bl/6J mice, P21.  
The model was fit based on perisomatic spiking activity and validated by simulating subthrehold somatic resonance.  
I$_h$ conductance is constant in the basal dendrites, increases exponentially with distance from the soma along the apical
trunk until the nexus with apical dendrite tufts, where the I$_h$ conductance plateaus at 0.006 S/cm$^2$ \citep{Harnett2015-sj}.  
% Allen cells
The last two mouse models \citep{Reimann2013-mg, Shai2015-ff, Markram2015-zg}
are based on data from primary visual cortex (V1) of C57BL/6 mice, P35-62. They based largely on a previous model also 
studied here \citep{Hay2011-if}.  The major modification to the original is that Ih conductance is constant in the apical and basal arbors.
This modification accounted for differences between dendritic sag, dendritic resting membrane potential relative 
to the soma, and dendritic input resistance compared to what is seen in rat L5 somatosensory cortex \citep{Shai2015-ff}.

\begin{table}[h!]
\begin{center}
    \begin{tabular}{m{2cm}|m{1.25cm}|m{2cm}|m{2.5cm}|m{1cm}|m{2cm}|m{3.75cm}}
        \hline
        \textbf{Model} & \textbf{Species} & \textbf{Strain} & \textbf{Region} & \textbf{Age} & \textbf{Max g$_{Ih}$ in dendrites (S/cm$^2$)} & \textbf{HCN Distribution}  \\
        \hline
        Hay et al. 2011 & Rat & Wistar & Neocortex & P36 & 0.015 & Constant in basal, exponential with distance in  apical\\
        \hline
        Acker \& Antic 2009 & Rat & Sprague Dawley & Frontal Cortex & P21-33 & 0.0025 & Constant in basal, exponential with distance in  apical\\
        \hline
        Kole et al. 2008 & Rat & Wistar & Somatosensory Cortex & P14-28 & 0.09 & Exponential with distance throughout dendritic arbor\\
        \hline
        Neymotin et al. 2017 & Mouse & C57Bl/6 & Primary Motor Cortex & P21 & 0.006 & Constant in basal, exponential with distance in  apical
        below nexus, constant after the nexus\\
        \hline
        ABA: 497233139 & Mouse & C57Bl/6 & Primary Visual Cortex & P35-62 & 3.7x10$^{-6}$
        & Constant in basal and constant in apical\\
        \hline
        ABA: 497232419 & Mouse & C57Bl/6 & Primary Visual Cortex & P35-62 & 2.8x10$^{-6}$
        & Constant in basal and constant in apical\\
        \hline
    \end{tabular}
\end{center}
\caption{\emph{Basic Model Information}: Models are specified by either the publication in which they first occurred or their Allen Brain Atlas 
(ABA) model identification number.  Ages are specified by postnatal day age. Under the comments on HCS distribution, "exponential with distance"
is with  respect to the soma.  For the ABA models, HCN distribution is constant in both apical and basal arbors, but different between the two.
In both models ABA, I$_h$ conductance is higher in the apical arbor than in the basal, but it is highest in the soma is 8.3x10$^{-6}$ S/cm$^2$ and 
8.1x10$^{-6}$ S/cm$^2$, respectively.}
\label{Table 1}
\end{table}

% simulations
Each dendritic segment was stimulated with a subthreshold chirp-waveform current injection, and membrane potential
was recorded from the site of stimulation and at the soma. For generating impednace profiles, the chirp-waveform 
instantaneous frequency changed linearly and spanned 0.5-50 Hz over 50s. To investigate bimodal phase responses, the
span of the chirp-waveform instantaneous frequency was increased to 0.5-100 Hz over a 100s duration.  By contrast,
most experimental studies of impedance limit the instantaneous frequency of the chirp stimulus to 20 Hz 
\citep{Ulrich2002-dd, Dembrow2015-zb}.  We computed Z$_{in}$ and Z$_c$ and associated measures from each of the 
recorded membrane potential waveforms via \emph{Eqs. 1-3}.  We focus our analysis on six measures of the complex 
impedance response: (1) $|$Z$_{in}|$ resonance amplitude, (2) $|$Z$_{in}|$ resonance frequency,
(3) $|$Z$_{in}|$ Q-factor, (4) $|$Z$_{c}|$ transfer frequency, (5) $\Phi_{in}$ leading phase bandwidth, 
and (6) $\Phi_{in}$ synchronous frequency (see \emph{Introduction}).

To study the influence of morphology, we performed the same chirp stimulation on all segments of cell models with
different morphologies but the same biophysical template \citep{Hay2011-if}.  Meanwhile, to study the roles of I$_h$ 
and I$_m$, we took a two pronged approach.  First, either I$_h$ or I$_m$ conductance was changed by $\pm$15\% its original
values across the dendritic arbor and generated impedance profiles as outlined above.  We then varied I$_h$ and I$_m$ in 5\%
increments up to $\pm$20\% at either the branch level (one apical and one basal) or whole-arbor level and computed the six 
complex impedance measures outlined above for an apical and a basal segment of the denritic arbor.

\section{Article types}

For requirements for a specific article type please refer to the Article Types on any Frontiers journal page. Please also refer to  \href{http://home.frontiersin.org/about/author-guidelines#Sections}{Author Guidelines} for further information on how to organize your manuscript in the required sections or their equivalents for your field

% For Original Research articles, please note that the Material and Methods section can be placed in any of the following ways: before Results, before Discussion or after Discussion.

\section{Manuscript Formatting}

\subsection{Heading Levels}

%There are 5 heading levels

\subsection{Level 2}
\subsubsection{Level 3}
\paragraph{Level 4}
\subparagraph{Level 5}


\subsection{Figures}
Frontiers requires figures to be submitted individually, in the same order as they are referred to in the manuscript. Figures will then be automatically embedded at the bottom of the submitted manuscript. Kindly ensure that each table and figure is mentioned in the text and in numerical order. Figures must be of sufficient resolution for publication \href{http://home.frontiersin.org/about/author-guidelines#ResolutionRequirements}{see here for examples and minimum requirements}. Figures which are not according to the guidelines will cause substantial delay during the production process. Please see \href{http://home.frontiersin.org/about/author-guidelines#GeneralStyleGuidelinesforFigures}{here} for full figure guidelines. Cite figures with subfigures as figure \ref{fig:2}B.


\subsubsection{Permission to Reuse and Copyright}
Figures, tables, and images will be published under a Creative Commons CC-BY licence and permission must be obtained for use of copyrighted material from other sources (including re-published/adapted/modified/partial figures and images from the internet). It is the responsibility of the authors to acquire the licenses, to follow any citation instructions requested by third-party rights holders, and cover any supplementary charges.
%%Figures, tables, and images will be published under a Creative Commons CC-BY licence and permission must be obtained for use of copyrighted material from other sources (including re-published/adapted/modified/partial figures and images from the internet). It is the responsibility of the authors to acquire the licenses, to follow any citation instructions requested by third-party rights holders, and cover any supplementary charges.

\subsection{Tables}
Tables should be inserted at the end of the manuscript. Please build your table directly in LaTeX.Tables provided as jpeg/tiff files will not be accepted. Please note that very large tables (covering several pages) cannot be included in the final PDF for reasons of space. These tables will be published as \href{http://home.frontiersin.org/about/author-guidelines#SupplementaryMaterial}{Supplementary Material} on the online article page at the time of acceptance. The author will be notified during the typesetting of the final article if this is the case. 

\section{Nomenclature}

\subsection{Resource Identification Initiative}
To take part in the Resource Identification Initiative, please use the corresponding catalog number and RRID in your current manuscript. For more information about the project and for steps on how to search for an RRID, please click \href{http://www.frontiersin.org/files/pdf/letter_to_author.pdf}{here}.

\subsection{Life Science Identifiers}
Life Science Identifiers (LSIDs) for ZOOBANK registered names or nomenclatural acts should be listed in the manuscript before the keywords. For more information on LSIDs please see \href{http://www.frontiersin.org/about/AuthorGuidelines#InclusionofZoologicalNomenclature}{Inclusion of Zoological Nomenclature} section of the guidelines.


\section{Additional Requirements}

For additional requirements for specific article types and further information please refer to \href{http://www.frontiersin.org/about/AuthorGuidelines#AdditionalRequirements}{Author Guidelines}.

\section*{Conflict of Interest Statement}
%All financial, commercial or other relationships that might be perceived by the academic community as representing a potential conflict of interest must be disclosed. If no such relationship exists, authors will be asked to confirm the following statement: 

The authors declare that the research was conducted in the absence of any commercial or financial relationships that could be construed as a potential conflict of interest.

\section*{Author Contributions}

The Author Contributions section is mandatory for all articles, including articles by sole authors. If an appropriate statement is not provided on submission, a standard one will be inserted during the production process. The Author Contributions statement must describe the contributions of individual authors referred to by their initials and, in doing so, all authors agree to be accountable for the content of the work. Please see  \href{http://home.frontiersin.org/about/author-guidelines#AuthorandContributors}{here} for full authorship criteria.

\section*{Funding}
Details of all funding sources should be provided, including grant numbers if applicable. Please ensure to add all necessary funding information, as after publication this is no longer possible.

\section*{Acknowledgments}
This is a short text to acknowledge the contributions of specific colleagues, institutions, or agencies that aided the efforts of the authors.

\section*{Supplemental Data}
 \href{http://home.frontiersin.org/about/author-guidelines#SupplementaryMaterial}{Supplementary Material} should be uploaded separately on submission, if there are Supplementary Figures, please include the caption in the same file as the figure. LaTeX Supplementary Material templates can be found in the Frontiers LaTeX folder.

\section*{Data Availability Statement}
The datasets [GENERATED/ANALYZED] for this study can be found in the [NAME OF REPOSITORY] [LINK].
% Please see the availability of data guidelines for more information, at https://www.frontiersin.org/about/author-guidelines#AvailabilityofData

\bibliographystyle{frontiersinSCNS_ENG_HUMS} % for Science, Engineering and Humanities and Social Sciences articles, for Humanities and Social Sciences articles please include page numbers in the in-text citations
%\bibliographystyle{frontiersinHLTH&FPHY} % for Health, Physics and Mathematics articles
\bibliography{impedance_biblio}

%%% Make sure to upload the bib file along with the tex file and PDF
%%% Please see the test.bib file for some examples of references

\section*{Figure captions}

%%% Please be aware that for original research articles we only permit a combined number of 15 figures and tables, one figure with multiple subfigures will count as only one figure.
%%% Use this if adding the figures directly in the mansucript, if so, please remember to also upload the files when submitting your article
%%% There is no need for adding the file termination, as long as you indicate where the file is saved. In the examples below the files (logo1.eps and logos.eps) are in the Frontiers LaTeX folder
%%% If using *.tif files convert them to .jpg or .png
%%%  NB logo1.eps is required in the path in order to correctly compile front page header %%%

%%%\begin{figure}[h!]
%%%\begin{center}
%%%\includegraphics[width=10cm]{logo1}% This is a *.eps file
%%%\end{center}
%%%\caption{ Enter the caption for your figure here.  Repeat as  necessary for each of your figures}\label{fig:1}
%%%\end{figure}


%%%\begin{figure}[h!]
%%%\begin{center}
%%%\includegraphics[width=15cm]{logos}
%%%\end{center}
%%%\caption{This is a figure with sub figures, \textbf{(A)} is one logo, \textbf{(B)} is a different logo.}\label{fig:2}
%%%\end{figure}

%%% If you are submitting a figure with subfigures please combine these into one image file with part labels integrated.
%%% If you don't add the figures in the LaTeX files, please upload them when submitting the article.
%%% Frontiers will add the figures at the end of the provisional pdf automatically
%%% The use of LaTeX coding to draw Diagrams/Figures/Structures should be avoided. They should be external callouts including graphics.

\end{document}